\subsection{Revisão e Trabalhos Relacionados}
\begin{frame}
\frametitle{Complexidade}
\framesubtitle{NP-Completo}
 \begin{columns}[T]
    \begin{column}{.5\textwidth}
    \centering
    \resizebox{\textwidth}{!}{%
        \centering
        \includegraphics[scale=.5]{img/rubik.jpeg}
    }
    \end{column}
    \begin{column}{.5\textwidth}
        NP-Completo:
        \begin{itemize}
            \item{Determinar se um grafo $G$ tem um conjunto envoltório de tamanho $k$}
            \item{Determinar se um grafo $G$ tem um conjunto Carathéodory de tamanho $k$}
        \end{itemize}
    \end{column}
  \end{columns}
\end{frame}

\begin{frame}
\frametitle{Complexidade}
\framesubtitle{NP-Completo}
 \begin{columns}[T]
    \begin{column}{.5\textwidth}
    \centering
    \resizebox{\textwidth}{!}{%
        \centering
        \includegraphics[scale=.5]{img/rubik.jpeg}
    }
    \end{column}
    \begin{column}{.5\textwidth}
        Explorar problemas NP-Completo:
        \begin{itemize}
            \item{Isolar e estudar casos polinomiais}
            \item{Estabelecer limites teóricos}
            \item{Soluções exponenciais para pequenas entradas}
            \item{Soluções aproximadas}
        \end{itemize}
    \end{column}
  \end{columns}
\end{frame}


\begin{frame}
\frametitle{Revisão Bibliográfica}
\framesubtitle{Resultados Nº Envoltório}
\begin{table}[H]
\centering
\label{tab-resultado-envoltoria}
    \begin{tabular}{c|c}
    \textbf{Resultado}    & \textbf{Grafo}  \\ \hline
    NP-Completude & Gerais (Redução SAT) e Planares $3 \le \Delta \le 4$ \\ \hline
    Limite  & \begin{tabular}[c]{@{}c@{}} Cografo, Cúbico, Tree-cograph, \\ Determinados produtos (KxK, PxK, CxK, GxH, PxC), \\ Produto Forte GxH, Produto Lexicografico GxH \\  \end{tabular} \\ \hline
    Algoritmo polinomial & \begin{tabular}[c]{@{}c@{}} Árvore, Cordal,  Cúbico, P4-Reducible, Bounded Treewidth \\ Bounded Rankwidth \end{tabular}  \\ \hline
    \end{tabular}
\end{table}
\end{frame}


\begin{frame}
\frametitle{Revisão Bibliográfica}
\framesubtitle{Resultados Nº Carathéodory}
\begin{table}[H]
    \centering
    \label{tab-resultado-caratheodory}
    \begin{tabular}{c|c}
        \textbf{Resultado}    & \textbf{Grafo} \\ \hline
        NP-Completude      & Bipartido (Redução 3-SAT)  \\ \hline
        Limite             &  \begin{tabular}[c]{@{}c@{}} Geral, Árvore, Bloco Especial \\ Claw-free, Cografos, Split  e P4-Reducible \end{tabular} \\ \hline
        Algoritmo Polinomial & \begin{tabular}[c]{@{}c@{}} Árvore, Chordal, Three-cograph \\ Bounded Treewidth e Rankwidth \end{tabular} \\ \hline
    \end{tabular}
\end{table}
\end{frame}