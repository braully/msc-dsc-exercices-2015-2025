\chapter{Introdução}
\label{cap:intro}

Um grafo sucintamente pode ser definido como um conjunto de elementos chamado de vértices, e por um conjunto de pares de vértices denominado arestas. Diversos problemas e relações do mundo real são modelados em grafos,
compondo uma ferramenta poderosa para o estudo e resolução de problemas. Embora vários problemas em grafos ainda sejam comprovadamente tidos como computacionalmente intratáveis (NP-Completo), como por exemplo, cacheiro viajante\cite{Biggs1986}, ainda assim são temas de interesse e ampla pesquisa. Em teoria dos grafos, como em outras disciplinas teóricas, alguns problemas são estudados e muitas vezes são resolvidos sem visar necessariamente a aplicação real e imediata. Na história da ciência moderna temos diversos exemplos, tais como a relatividade restrita e a mecânica quântica, que antes não tinham aplicações quando pesquisadas e hoje nos proporciona o gozar de tecnologias que vão do GPS à tomografia. Problemas como o Bóson de Higgs\cite{Higgs1964} foram estudados e resolvidos.Ainda permanecem sem aplicações práticas, porém cumprem seu papel preenchendo as lacunas na ciência.

Um caso interessante é o estudo da convexidade em grafos. Esse tópico foi motivado pelo estudo da convexidade geométrica na matemática. O início dos estudos nesta área data de aproximadamente 300 a.C
com os primeiros trabalhos de Euclides (300 a.C.) e seus polígonos convexos, mas somente a partir dos anos 30 se tem o início de estudos sistemáticos voltado á aplicações e convexidade de conjuntos.
Uma potencial e interessante aplicação do arcabouço teórico da convexidade
é o estudo do modelo de contaminação social ou disseminação de infecções \cite{Hodas2014}. Um indivíduo em uma rede se torna potencialmente contaminado a medida que seus vizinhos na rede ficam contaminados. Em publicação anterior \cite{Dreyer2009}, onde podemos ver um modelo matemático relativamente simples, baseado em grafos, que possibilita fomentar tais estudos. O modelo descreve os indivíduos por vértices e suas relações como arestas. Os vértices entram em um estado irreversível $X$ caso uma quantidade miníma $k$ de seus vizinhos também entrar nesse estado. Em \cite{Kuhlman2011} temos um estudo analítico e experimental de contaminação seguindo o modelo apresentado em \cite{Dreyer2009} limitado a 2 vizinhos. Esse modelo é compatível com os estudos de convexidade $P_3$ em grafos.

Na convexidade em grafos, podemos definir quando um subconjunto de vértices $S$ é um conjunto convexo. Na convexidade $P_3$ um subconjunto de vértices é convexo quando todo vértice tendo dois vizinhos em $S$, também estão em $S$. Dentro do universo de estudo de convexidades em grafos este trabalho realizou estudos e implementações pertinentes a dois parâmetros na convexidade $P_3$, a saber: número envoltório e número de Carathéodory na convexidade $P_3$.

O conceito do número envoltório foi introduzido por Everett e Seidman \cite{everett1985hull} na convexidade geodética e considera o conceito de envoltória convexa. Dado um conjunto $S$ de vértices, a envoltória convexa de $S$ é o menor conjunto convexo contendo $S$. E o número envoltório é a cardinalidade do menor conjunto $S$ tal que a envoltória convexa desse conjunto seja o conjunto de vértices do grafo. 
A determinação do número envoltório na convexidade $P_3$ é um problema NP-difícil, mas pode ser determinado em tempo polinomial para algumas classes, como cografos e grafos cordais \cite{Centeno}, e para prismas complementares \cite{duarte2015complexity}.

Recentemente, Penso Et. al\cite{Penso2015} estudaram a complexidade de calcular o número envoltório restrito a grafos planares com grau máximo limitado, mostrando que o problema continua NP-completo para grafos planares com grau máximo 3 e 4. Neste trabalho, estudamos o número envoltório considerando o diâmetro do grafo. Em particular, apresentamos um limite superior do número Envoltório $P_3$ para grafos de diâmetro 2 e também limites específicos para as subclasses maximal livre de triângulo e fortemente regulares. Os nossos resultados para grafos de diâmetro 2 e maximais livres de triângulo foram aceitos no Cologne-Twente Workshop on Graphs and Combinatorial Optimization 2018 - CTW 2018 e os limites para grafos fortemente regulares foram apresentados no VIII Latin American Workshop on Cliques in Graphs - LAWCG 2018 \cite{Coelho2018}.

Outro parâmetro bem conhecido sobre os conjuntos convexos é o número de Carathéodory. Este parâmetro surge do teorema de Carathéodory para conjuntos convexos no ${\cal R}^d$ \cite{Caratheodory1911}.
Com base neste teorema todo ponto $u$ na envoltória convexa de um conjunto $S \subseteq {\cal R}^d$ encontra-se no fecho
convexo de um subconjunto $F$ de $S$ de ordem no máximo $d+1$. Do teorema de Carathéodory surge  a definição do número de 
Carathéodory para grafos. Seja $G$ um grafo e ${\cal C}$ uma convexidade sobre V(G), o \textit{número de Carathéodory} c(G) é o menor inteiro c, para o qual todo $u \in H(S)$, existe um conjunto $F \subseteq  S$, com $|F| \le c$ e $u \in H(F)$.  Seja $G$ um grafo e $S$ um subconjunto de V(G). Se $\partial H(S)=H(S) \setminus \bigcup _{u \in S} H(S \setminus \{u\})$, é não vazio, então $S$ é um \textit{conjunto de Carathéodory}. Esta definição permite uma forma alternativa para o número de Carathéodory como sendo a maior cardinalidade de um conjunto de Carathéodory. Na convexidade $P_3$ a determinação do número de Carathéodory é um problema NP-completo, mesmo para grafos bipartidos cite{Barbosa2012}, mas possui determinação de tempo polinomial para árvores, grafos blocos \cite{Barbosa2012} e cordais \cite{Coelho2014}. Neste trabalho, consideramos o número de Carathéodory na convexidade $P_3$ para grafos de diâmetro 2 com vértice de corte e algumas subclasses dos grafos maximais livres de triângulo.

Ainda implementamos vários algoritmos. Desenvolvemos algoritmos de força bruta e algoritmo heurísticos para o número envoltório e número de Carathéodory. Como esses problemas são NP-completos, implementamos versões paralelas de todos eles com o objetivo de melhorar o tempo de execução. As implementações realizadas foram fundamentais para a escolha de trabalhar com grafos de diâmetro 2 e para as conjecturas levantadas e posteriormente provadas.

Com os algoritmos implementados desenvolvemos uma ferramenta simplificada, denominada {\it FATIG}, com facilidades para uso e novas implementações em outros tipos de problemas envolvendo grafos.

Por fim, tratamos dos grafos de Moore de diâmetro 2, 
que é um resultado colateral identificado ao longo deste trabalho. Os grafos de Moore são grafos de diâmetro 2 que atendem a igualdade que estabelecemos para os grafos maximais livre de triângulo. Foi desenvolvido um algoritmo capaz de gerar todos os três grafos de Moore, e este algoritmo pode ser uma ferramenta auxiliar para encontrar o último grafo de Moore. O último grafo de Moore, em suposição, é um grafo 57-regular com 3250 vértices. Sua existência ainda permanece um problema em aberto, mas vários de seus parâmetros e propriedades já foram identificados em outros trabalhos \cite{Godsil1995,Macaj2010,Miller2005}.

Este trabalho está dividido em 5 capítulos a saber: No capítulo \ref{cap-fundamentos} apresentamos os conceito básicos e notações utilizadas no trabalho, fazemos também uma revisão dos resultados conhecidos sobre o tema. No capítulo \ref{envoltoria} apresentamos nossos resultados sobre o número envoltório para grafos de diâmetro 2.
O capítulo \ref{cara} é dedicado aos resultados sobre o número de Carathéodory para alguns grafos de diâmetro 2.% sem $C_6$ com subgrafo induzido. 
 A discussão sobre os grafos de Moore estão no Capítulo \ref{moore} e as implementações dos algoritmos são apresentadas no Capítulo \ref{implementa}. Finalizamos o trabalho com nossas conclusões e propostas á futuros trabalhos.

