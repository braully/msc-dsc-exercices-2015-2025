\subsection{Minimal Ramsey}
/* Definir a classe, apontar as referência, comentar os resultados */
\noindent\begin{minipage}[t]{.35\textwidth}
\captionof{figure}{Comportamento c(G) MR.}
\label{graf-comportamento-minimal-ramsey}
\centering
\resizebox{.95\textwidth}{!}{%
    \begin{tikzpicture}
        \begin{axis}
            [xlabel={Nº de Vértices}, ylabel={Nº de Carathéodory}, 
             legend pos=north west,clip=false,axis lines=left,
             ymin=1, ymax=5, ytick={2,3,4,5},
             xtick={2,6,10,18,19,30}]
             \addplot[name path=maxi,color=red,mark=none] coordinates {
                (3,2) (30,2) 
             };
             \addlegendentry{mínimo}

             \addplot[name path=mini,color=blue,mark=none] coordinates {
                (3,2) (6,3) (10,4)
                (18,4) (19,3)
                (30,3) 
             };
             \addlegendentry{máximo}
             \addplot[fill=blue, fill opacity=0.3] fill between[of=maxi and mini];
        \end{axis}
    \end{tikzpicture}%
}
\end{minipage}
\begin{minipage}[t]{.65\textwidth}
\captionof{table}{Comparativo algoritmos MR.}
\label{tab-comparativo-minimal-ramsey}
\centering
\resizebox{.95\textwidth}{!}{%
    \begin{tabular}{r|r|r|r|r|r|r}
    \textbf{} & \multicolumn{3}{c|}{\textbf{\begin{tabular}[c]{@{}c@{}}Algoritmo~\ref{alg:numero-caratheodory-p3} \\ $NumeroCaratheodory_{cp3}$\end{tabular}}} 
              & \multicolumn{3}{c}{\textbf{\begin{tabular}[c]{@{}c@{}}Algoritmo~\ref{alg:aproximativo-numero-caratheodory-p3} \\ $AproxNCaratheodory_{cp3}$\end{tabular}}} \\ \hline
    \textbf{Nº Vert.} & \textbf{Quantidade} & \textbf{c(g)} & \textbf{T(s)} & \textbf{Acertos} & \textbf{max$\{\Delta(r)\}$} & \textbf{T(s)} \\ \hline
	20 & 15   & {[}3,4{]} & 1,47      & 6,7\%   & 1 & 0,15 \\
	22 & 1    & 3         & 0,3       & 100,0\% & 0 & 0,21 \\
	23 & 106  & {[}4,8{]} & 94,69     & 12,3\%  & 1 & 0,97 \\
	24 & 53   & {[}4,6{]} & 59,67     & 1,9\%   & 1 & 0,47 \\
	27 & 700  & {[}4{]}   & 4527,28   & 27,4\%  & 1 & 9,06 \\
	30 & 5084 & {[}3,5{]} & 105436,84 & 35,2\%  & 1 & 159,73 \\
	32 & 1013 & {[}3,4{]} & 63284,24  & 3,8\%   & 1 & 39,0
    \end{tabular}%
}
\end{minipage}



Para o número envoltório temos os seguintes resultados.


\noindent\begin{minipage}[t]{.35\textwidth}
\captionof{figure}{Comportamento h(G) MR.}
\label{graf-comportamento-hn-mr}
\centering
\resizebox{.95\textwidth}{!}{%
    \begin{tikzpicture}
        \begin{axis}
            [xlabel={Nº de Vértices}, ylabel={Nº de Carathéodory}, 
             legend pos=north west,clip=false,axis lines=left,
             ymin=1, ytick={2,3,4,5,6,7,8,9}, xmax=33,
             xtick={16,18,20,22,24,26,28,30,32}]
             \addplot[name path=maxi,color=red,mark=none] coordinates {
                (16,6) (17,5) (18,4)
                (19,1.98) (32,1.98)
             };
             \addplot[name path=mini,color=blue,mark=none] coordinates {
                (16,7) (17,6) (18,4)
                (19,7) (20,6) (21,4) (22,3)
                (23,7) (24,6) (26,6)
                (27,3) (28,2) (29,6) (30,5)
                (31,2) (32,2)
             };
             \addlegendentry{mínimo}
             \addlegendentry{máximo}
             \addplot[fill=blue, fill opacity=0.3] fill between[of=maxi and mini];
        \end{axis}
    \end{tikzpicture}%
}
\end{minipage}
\begin{minipage}[t]{.65\textwidth}
\captionof{table}{Comparativo algoritmos MR}
\label{tab-comparativo-hn-mr}
\centering
\resizebox{.95\textwidth}{!}{%
    \begin{tabular}{r|r|r|r|r|r|r}
    \textbf{} & \multicolumn{3}{c|}{\textbf{\begin{tabular}[c]{@{}c@{}}Algoritmo~\ref{alg:numero-envoltoria-p3} \\ $NumeroEnvoltorio_{cp3}$\end{tabular}}} 
              & \multicolumn{3}{c}{\textbf{\begin{tabular}[c]{@{}c@{}}Algoritmo~\ref{alg:aproximativo-numero-envoltoria-p3} \\ $AproxNEnvoltoria_{cp3}$\end{tabular}}} \\ \hline
    \textbf{Nº Vert.} & \textbf{Quantidade} & \textbf{h(g)} & \textbf{T(s)} & \textbf{Acertos} & \textbf{max$\{\Delta(r)\}$} & \textbf{T(s)} \\ \hline
	16 & 2    & {[}6,7{]} & 0,02 & 100,0\% & 0 & 0     \\
	17 & 2    & {[}5,6{]} & 0,02 & 100,0\% & 0 & 0     \\
	18 & 1    & {[}4,4{]} & 0    & 100,0\% & 0 & 0     \\
	19 & 13   & {[}2,7{]} & 0,11 & 38,5\%  & 1 & 0,01  \\
	20 & 18   & {[}2,6{]} & 0,11 & 88,9\%  & 1 & 0,01  \\
	21 & 5    & {[}2,4{]} & 0    & 100,0\% & 0 & 0     \\
	22 & 22   & {[}2,3{]} & 0    & 81,8\%  & 1 & 0,02  \\
	23 & 106  & {[}2,7{]} & 1,12 & 59,4\%  & 1 & 0,1   \\
	24 & 53   & {[}2,6{]} & 0,09 & 83,0\%  & 1 & 0,07  \\
	25 & 397  & {[}2,6{]} & 0,11 & 100,0\% & 0 & 0,57  \\
	26 & 63   & {[}2,6{]} & 0,14 & 100,0\% & 0 & 0,1   \\
	27 & 704  & {[}2,3{]} & 0,01 & 74,7\%  & 1 & 1,34  \\
	28 & 126  & 2 & 0    & 100,0\% & 0 & 0,24  \\
	29 & 1347 & {[}2,6{]} & 0,73 & 99,9\%  & 1 & 3,12  \\
	30 & 6884 & {[}2,5{]} & 0,24 & 74,5\%  & 1 & 16,87 \\
	31 & 3217 & 2 & 0,06 & 88,2\%  & 1 & 9,07  \\
	32 & 6631 & 2 & 0,1  & 99,2\%  & 1 & 23,64
    \end{tabular}%
}
\end{minipage}

\subsection{Ramsey numbers}
/* Definir a classe, apontar as referência, comentar os resultados */



\subsection{Hipo-hamiltoniano}
Um grafo $G$ é \textit{hipo-hamiltoniano} se não é hamiltoniano,
porém com a remoção de qualquer vértice, torna o grafo $G$ hamiltoniano.
Na coletânea de grafos \textit{Hipo-hamiltoniano} em \cite{hog2013},
temos todos grafos possíveis desse tipo de 10 até 19 vértices \cite{Aldred1997,Goedgebeur2016},
e mais alguns encontrados de 20 até 24 vértices pelo gerador desenvolvido por \cite{Goedgebeur2016a}.

Em analise prévia conforme gráfico da Tabela~\ref{tab-comportamento-ahh},
vemos que o crescimento aparente do número de Carathéodory $P_3$
para esse tipo de grafo é ilimitado, e com uma tendência
de crescimento não previsível, 
indicando a dificuldade de atribuir um limite justo para
esse parâmetro nesse tipo de grafo. 


\begin{multicols}{2}
%\begin{multicols}{2}
\raggedleft
\resizebox{.60\textwidth}{!}{%
    \begin{tabular}{r|r|r|r|r|r|r}
    \textbf{} & \multicolumn{3}{c|}{\textbf{\begin{tabular}[c]{@{}c@{}}Algoritmo 3.\ref{alg:numero-caratheodory-p3} \\ $NumeroCaratheodory_{cp3}$\end{tabular}}} 
              & \multicolumn{3}{c}{\textbf{\begin{tabular}[c]{@{}c@{}}Algoritmo 3.\ref{alg:aproximativo-numero-caratheodory-p3} \\ $AproxNCaratheodory_{cp3}$\end{tabular}}} \\ \hline
    \textbf{Nº Vert.} & \textbf{Quantidade} & \textbf{c(g)} & \textbf{T(s)} & \textbf{Acertos} & \textbf{max$\{\Delta(r)\}$} & \textbf{T(s)} \\ \hline
    10,13,15         & 8 (100\%)           & 4             & 0,30            & 25,0\%             & 1                         & 0,01             \\
    16               & 12 (100\%)          & {[}4,5{]}     & 0,62            & 75,0\%             & 1                         & 0,03             \\
    18               & 35 (100\%)          & {[}4,6{]}     & 4,93            & 28,0\%             & 1                         & 0,01              \\
    19               & 102 (100\%)         & {[}5,6{]}     & 37,8            & 14,7\%             & 2                         & 0,36              \\
    20               & 102 (amostras)                 & {[}4,7{]}     & 80,5            & 29,4\%             & 2                         & 0,36              \\    
    21               & 85 (amostras)                 & {[}5,7{]}     & 109,7           & 32,9\%             & 2                         & 0,4              \\
    22               & 420 (amostras)                 & {[}5,8{]}     & 1101,49         & 17,9\%             & 2                         & 2,23              \\
    23               & 85 (amostras)                  & {[}6,7{]}     & 522,04          & 30,6\%             & 1                         & 0,53              \\
    24               & 2530 (amostras)                & {[}6,8{]}     & 30887,49        & 10,5\%             & 2                         & 17,56             
    \end{tabular}
}
\captionof{table}{Resultados c(G) hipo-hamiltoniano.}\label{tab-comportamento-ahh}
\columnbreak

\resizebox{.35\textwidth}{!}{%
    \begin{tikzpicture}
        \begin{axis}
            [xlabel={Nº de Vértices}, ylabel={Nº de Carathéodory}, 
             legend pos=north west,clip=false,axis lines=left,
             ymin=1, ytick={2,3,4,5,6,7,8,9}, xmin=14,
             xtick={15,16,18,19,20,21,22,23,24}]
             \addplot[name path=maxi,color=red,mark=none] coordinates {
                %(10,4) 
                (14,4) (15,4)
                (16,4) (18,4) (19,5)
                %(19,5) (20,4) 
                %(21,5) (22,5) 
                %(23,6) (24,5)
             };

             \addplot[name path=mini,color=blue,mark=none] coordinates {
                %(10,4) 
                (14,4) (15,4)
                (16,5) (18,6) (19,6)
                %(19,6) (20,7) 
                %(21,7) (22,8) 
                %(23,7) (24,8)
             };
             
             \addlegendentry{mínimo}
             \addlegendentry{máximo}
             \addplot[dotted, name path=miniaprox,color=red] coordinates {
                (19,5) (20,4) (21,5) (22,5) (23,6) (24,6)
             };
            \addplot[dotted, name path=maxaprox,color=blue] coordinates {
                (19,6) (20,7) (21,7) (22,8) (23,7) (24,8)
             };
             \addplot[fill=blue, fill opacity=0.3] fill between[of=maxi and mini];
        \end{axis}
    \end{tikzpicture}%
}

%\multicolinterrupt{}

%%\end{minipage}
%\columnbreak
%%\end{minipage}
\end{multicols}


Para o número envoltório também não temos uma tendência de resultado previsível para os limites.


\begin{multicols}{2}
%\captionof{figure}{Comportamento h(G) H.}
%\label{graf-comportamento-hn-hh}
\raggedleft
\resizebox{.60\textwidth}{!}{%
    \begin{tabular}{r|r|r|r|r|r|r}
    \textbf{} & \multicolumn{3}{c|}{\textbf{\begin{tabular}[c]{@{}c@{}}Algoritmo 3.\ref{alg:numero-envoltoria-p3} \\ $NumeroEnvoltorio_{cp3}$\end{tabular}}} 
              & \multicolumn{3}{c}{\textbf{\begin{tabular}[c]{@{}c@{}}Algoritmo 3.\ref{alg:aproximativo-numero-envoltoria-p3} \\ $AproxNEnvoltoria_{cp3}$\end{tabular}}} \\ \hline
    \textbf{Nº Vert.} & \textbf{Quantidade} & \textbf{h(g)} & \textbf{T(s)} & \textbf{Acertos} & \textbf{max$\{\Delta(r)\}$} & \textbf{T(s)} \\ \hline
%        10   & 3 (100\%)   & 3 & 0,01 & 0,0\%   & 1 & 0    \\
%        13   & 3    & 3 & 0    & 100,0\% & 0 & 0    \\
        10,13 & 6 (100\%)    & 3    & 0,01 & 50,0\%   & 1 & 0    \\
        15    & 2 (100\%)    & 4         & 0    & 100,0\% & 0 & 0    \\
        16    & 12 (100\%)   & {[}3,4{]} & 0,01 & 25,0\%  & 1 & 0,01 \\
        18,19 & 137 (100\%)  & {[}3,5{]} & 0,2  & 55,4\%  & 1 & 0,02 \\
%        18   & 35 (amostras)           & {[}3,5{]} & 0,1  & 37,1\%  & 1 & 0,02 \\
%        19   & 102 (amostras)          & {[}3,5{]} & 0,1  & 61,8\%  & 1 & 0,06 \\
        20    & 102 (amostras)          & {[}3,6{]} & 0,16 & 42,2\%  & 1 & 0,05 \\
        21    & 85 (amostras)           & {[}4,5{]} & 0,41 & 44,7\%  & 1 & 0,06 \\
        22    & 420 (amostras)          & {[}3,6{]} & 1,34 & 52,4\%  & 1 & 0,23 \\
        23    & 85 (amostras)           & {[}4,6{]} & 1,15 & 27,1\%  & 1 & 0,05 \\
        24    & 2.530 (amostras)        & {[}4,6{]} & 39,8 & 36,7\%  & 2 & 1,67
    \end{tabular}%
}

\captionof{table}{Resultados h(G) hipo.}\label{tab-comparativo-hn-hh}
\columnbreak

\resizebox{.35\textwidth}{!}{%
    \begin{tikzpicture}
        \begin{axis}
            [xlabel={Nº de Vértices}, ylabel={Nº Envoltório}, 
             legend pos=north west,clip=false,axis lines=left,
             ymin=1, ytick={2,3,4,5,6,7,8,9},
             xtick={10,13,15,16,18,19,20,21,22,23,24}]
             \addplot[name path=maxi,color=red,mark=none] coordinates {   
                (10,3) (13,3) (15,4)
                (16,4) (18,3) (19,3)
                %(20,3) (21,4) (22,3) 
                %(23,4) (24,4)
             };
             \addplot[name path=mini,color=blue,mark=none] coordinates {
                (10,3) (13,3) (15,4)
                (16,5) (18,6) (19,6)
                %(20,7) (21,7) (22,8) 
                %(23,7) (24,8)
             };
             \addplot[fill=blue, fill opacity=0.3] fill between[of=maxi and mini];
             \addlegendentry{mínimo}
             \addlegendentry{máximo}
             \addplot[dotted, name path=miniaprox,color=red] coordinates {
                (19,3)
                (20,3) (21,4) (22,3) 
                (23,4) (24,4)
             };
            \addplot[dotted, name path=maxaprox,color=blue] coordinates {
                (19,6)
                (20,7) (21,7) (22,8) 
                (23,7) (24,8)
             };
        \end{axis}
    \end{tikzpicture}%
}
%\captionof{table}{Resultados h(G) hipo-hamiltoniano}
\end{multicols}


Os algoritmos aproximativos do número envoltório e Carathéodory,
não apresentaram bons resultados para os grafos hipo-hamiltonianos,
com uma média de acertos exatos abaixo de 50\%.
E nos piores o resultado aproximado ficou duas unidades de distância do resultado esperado.

Para estes grafos não é possível obter conjecturas dos limites dos parâmetros
E como os algoritmos aproximativos não obtiveram bons resultados. 
Para estes grafos então não tivemos avanços nos estudos dos parâmetros 
com as implementações deste trabalho.

\subsection{Quase hipo-hamiltoniano}

Um grafo $G$ é \textit{quase hipo-hamiltoniano} se ele é não hamiltoniano, porém
existe um vértice v, tal que G-v é \textit{hipo-hamiltoniano}.
Na coletânea de grafos \textit{quase hipo-hamiltoniano} em \cite{hog2013},
traz aproximadamente 831 grafos, sendo o maior de 32 vértices,
baseado nos trabalhos de \cite{Goedgebeur2016},
apesar de não contemplar todos os grafos até 32 vértices.

É possível ver conforme o gráfico da Tabela~\ref{tab-comportamento-ahh} 
que o crescimento aparente do número de Carathéodory $P_3$
para esse tipo de grafo é ilimitado, 
e com uma tendência de crescimento não previsível, 
indicando a dificuldade de atribuir um limite justo para
esse parâmetro nesse tipo de grafo.

\begin{multicols}{2}
%\captionof{figure}{Comportamento c(G) A.H.}
%\label{graf-comportamento-ahh}
\centering
\resizebox{.60\textwidth}{!}{%
    \begin{tabular}{r|r|r|r|r|r|r}
    \textbf{} & \multicolumn{3}{c|}{\textbf{\begin{tabular}[c]{@{}c@{}}Algoritmo 3.\ref{alg:numero-caratheodory-p3} \\ $NumeroCaratheodory_{cp3}$\end{tabular}}} 
              & \multicolumn{3}{c}{\textbf{\begin{tabular}[c]{@{}c@{}}Algoritmo 3.\ref{alg:aproximativo-numero-caratheodory-p3} \\ $AproxNCaratheodory_{cp3}$\end{tabular}}} \\ \hline
    \textbf{Nº Vert.} & \textbf{Quantidade} & \textbf{c(g)} & \textbf{T(s)} & \textbf{Acertos} & \textbf{max$\{\Delta(r)\}$} & \textbf{T(s)} \\ \hline
    17     &6 (100\%)         & 5             & 1,69               & 50\%                 & 0                        & 0             \\
    18     &5 (100\%)         & {[}5,6{]}     & 1,19               & 50\%                 & 1                        & 0,6             \\
    19               &31 (amostras)      & {[}4,6{]}     & 16,26              & 28,6\%                 & 2                        & 0,84             \\
    20               &14 (amostras)      & {[}5,6{]}     & 14,72              & 100\%                 & 1                        & 0,93             \\
    21               &27 (amostras)      & 6             & 51,56              & 32,7\%                 & 1                        & 0,78             \\
    22               &133 (amostras)     & {[}6,7{]}     & 403,1              & 16,7\%                 & 2                        & 0,88             \\
    23               &404 (amostras)     & {[}5,7{]}     & 2446,66            & 20\%                 & 2                        & 0,79             \\
    24               &68 (amostras)      & {[}6,8{]}     & 800,93             & 10,5\%                 & 2                        & 1,03             \\
    \end{tabular}%
}
\captionof{table}{Resultados c(G) quase hipo-hamiltoniano}
%\label{tab-comparativo-ahh}

%\end{minipage}
\columnbreak
%\captionof{table}{Comparativo algoritmos A.H.}
%\label{tab-comparativo-ahh}
%\centering
\raggedleft
\resizebox{.35\textwidth}{!}{%
    \begin{tikzpicture}
        \begin{axis}
            [xlabel={Nº de Vértices}, ylabel={Nº de Carathéodory}, 
             legend pos=north west,clip=false,axis lines=left,
             ymin=1, ytick={2,3,4,5,6,7,8,9},
             xtick={15,16,17,18,19,20,21,22,23,24,25}]
             \addplot[name path=maxi,color=red,mark=none] coordinates {
                (17,5) (18,5) 
                %(19,4) (20,5) (21,6) (22,6)
                %(23,5) (24,6) 
             };
             \addplot[name path=mini,color=blue,mark=none] coordinates {
                (17,5) (18,6)
                %(19,6) (20,6) (21,6) (22,7)
                %(23,7) (24,8) 
             };

             \addlegendentry{mínimo}
             \addlegendentry{máximo}
             \addplot[dotted, name path=miniaprox,color=red] coordinates {
                (18,5) (19,4) (20,5) (21,6) (22,6)
                (23,5) (24,6) 
             };
            \addplot[dotted, name path=maxaprox,color=blue] coordinates {
                (18,6) (19,6) (20,6) (21,6) (22,7)
                (23,7) (24,8) 
             };
             \addplot[fill=blue, fill opacity=0.3] fill between[of=maxi and mini];
        \end{axis}
    \end{tikzpicture}%
}
\end{multicols}


Para o número envoltório também não temos bons resultados,
para os limites da classe, 
conforme pode ser visto no gráfico da Tabela~\ref{tab-comparativo-hn-qhh}.

\begin{multicols}{2}
%\captionof{figure}{Comportamento h(G) AHH.}
%\label{graf-comportamento-hn-qhh}
%\centering
\raggedleft
\resizebox{.60\textwidth}{!}{%
    \begin{tabular}{r|r|r|r|r|r|r}
    \textbf{} & \multicolumn{3}{c|}{\textbf{\begin{tabular}[c]{@{}c@{}}Algoritmo 3.\ref{alg:numero-envoltoria-p3} \\ $NumeroEnvoltorio_{cp3}$\end{tabular}}} 
              & \multicolumn{3}{c}{\textbf{\begin{tabular}[c]{@{}c@{}}Algoritmo 3.\ref{alg:aproximativo-numero-envoltoria-p3} \\ $AproxNEnvoltoria_{cp3}$\end{tabular}}} \\ \hline
    \textbf{Nº Vert.} & \textbf{Quantidade} & \textbf{h(g)} & \textbf{T(s)} & \textbf{Acertos} & \textbf{max$\{\Delta(r)\}$} & \textbf{T(s)} \\ \hline
%	17 & 6   & 4 & 0,02    & 0,0\%   & 1 & 0,01 \\
%	18 & 5   & 4 & 0,02    & 0,0\%   & 1 & 0,01 \\
	17-18 (100\%) & 11 & 4 & 0,04    & 0,0\%   & 1 & 0,01 \\
	19-22 & 205 (amostras)  & {[}3,4{]} & 0,65    & 41,3\%  & 1 & 0,14 \\
%	19 & 31 (amostras)  & {[}3,4{]} & 0,04    & 61,3\%  & 1 & 0,02 \\
%	20 & 14 (amostras)  & {[}4,5{]} & 0,03    & 14,3\%  & 1 & 0,01 \\
%	21 & 27 (amostras)  & {[}4,5{]} & 0,12    & 25,9\%  & 1 & 0,02 \\
%	22 & 133 (amostras) & {[}4,5{]} & 0,46    & 41,4\%  & 1 & 0,09 \\
%	23 & 404 (amostras) & {[}4,6{]} & 2,43    & 37,9\%  & 1 & 0,23 \\
%	24 & 68 (amostras)  & {[}4,6{]} & 1,04    & 22,1\%  & 2 & 0,04 \\
	23-24 & 472 (amostras) & {[}4,6{]} & 3,47    & 35,9\%  & 2 & 0,23 \\
	26 & 20 (amostras)  & 7 & 6,83    & 20,0\%  & 1 & 0,05 \\
%	28 & 8 (amostras)   & 8 & 15,6    & 62,5\%  & 1 & 0,01 \\
%	30 & 37 (amostras)  & 8 & 135,89  & 5,4\%   & 1 & 0,06 \\
     28-30 &  45 (amostras) & 8 & 151,49    & 15,5\%  & 1 & 0,01 \\
	32 & 78 (amostras)  & 9 & 1796,36 & 60,3\%  & 2 & 0,12
    \end{tabular}%
}
\captionof{table}{Resultados h(G) quase hipo-hamiltoniano}
\label{tab-comparativo-hn-qhh}

%\end{minipage}
\columnbreak
%\captionof{table}{Comparativo algoritmos AHH}
%\label{tab-comparativo-hn-qhh}
%\centering
%\raggedright
\resizebox{.35\textwidth}{!}{%
    \begin{tikzpicture}
        \begin{axis}
            [xlabel={Nº de Vértices}, ylabel={Nº Envoltório}, 
             legend pos=north west,clip=false,axis lines=left,
             ymin=1, ytick={2,3,4,5,6,7,8,9},
             xtick={17,18,19,20,21,22,23,24,26,28,30,32}]
             \addplot[name path=maxi,color=red,mark=none] coordinates {
                (17,4) (18,4) 
             };
             \addplot[name path=mini,color=blue,mark=none] coordinates {
                (17,4) (18,4) 
             };
             \addplot[fill=blue, fill opacity=0.3] fill between[of=maxi and mini];

             \addlegendentry{mínimo}
             \addlegendentry{máximo}
             \addplot[dotted, name path=miniaprox,color=red] coordinates {
                (18,4) (19,3) (22,3) (23,4) (24,4) (26,7) (28,8) (30,8) (32,9) 
             };
            \addplot[dotted, name path=maxaprox,color=blue] coordinates {
                (18,4) (19,4) (22,4) (23,6) (24,6) (26,7) (28,8) (30,8) (32,9) 
             };
        \end{axis}
    \end{tikzpicture}%
}
\end{multicols}

Assim como para os hipo-hamiltonianos, 
os algoritmos aproximativos do número envoltório e Carathéodory,
não apresentaram bons resultados para os grafos \textit{quase hipo-hamiltonianos},
com uma média de acertos exatos abaixo de 50\%.
E nos piores o resultado aproximado ficou duas unidades de distância do resultado esperado.

Para estes grafos não foi possível obter conjecturas dos limites dos parâmetros.
E como os algoritmos aproximativos não obtiveram bons resultados. 
Para estes grafos então não tivemos avanços nos estudos dos parâmetros 
com as implementações deste trabalho.



\subsection{Cúbico}
%/* Definir a classe, apontar as referência, comentar os resultados */

Um grafo $G$ é \textit{cúbico} se todo os vértices tem grau 3.
A coletânea obtida em \cite{hog2013},
é resultante dos trabalhos de \cite{Brinkmann2011,Brinkmann1995,Robinson1983},
todos os grafos cúbicos de 6 até 24 vértices, livres de isomorfismo, estão disponíveis,
de 26 até 64 existe uma volumosa quantidade de grafos,
mesmo que nem todos estejam contemplados.

É possível ver conforme o gráfico da Tabela~\ref{tab-comparativo-cubic} 
que o crescimento aparente do número de Carathéodory $P_3$
para esse tipo de grafo tem uma tendência de crescimento ilimitado e pouco previsível.

\begin{multicols}{2}
%\captionof{figure}{Comportamento c(G) Cubico.}
%\label{graf-comportamento-cubic}
\raggedleft
\resizebox{.60\textwidth}{!}{%
    \begin{tabular}{r|r|r|r|r|r|r}
    \textbf{} & \multicolumn{3}{c|}{\textbf{\begin{tabular}[c]{@{}c@{}}Algoritmo 3.\ref{alg:numero-caratheodory-p3} \\ $NumeroCaratheodory$\end{tabular}}} 
              & \multicolumn{3}{c}{\textbf{\begin{tabular}[c]{@{}c@{}}Algoritmo 3.\ref{alg:aproximativo-numero-caratheodory-p3} \\ $AproxNCaratheodory$\end{tabular}}} \\ \hline
    \textbf{Nº Vert.} & \textbf{Quantidade} & \textbf{c(G)} & \textbf{T(s)} & \textbf{Acertos} & \textbf{max$\{\Delta(r)\}$} & \textbf{T(s)} \\ \hline
        4  & 1 (100,0\%)      & 2   & 0,01      & 100,0\% & 0 & 0        \\
        6  & 2 (100,0\%)      & 2   & 0,01      & 100,0\% & 0 & 0,02     \\
        8  & 5 (100,0\%)      & 3   & 0,03      & 100,0\% & 0 & 0,04     \\
        10,12 & 104 (100,0\%)     & {[}3,4{]}   & 0,42      & 96,5\%  & 1 & 0,27     \\
%        12 & 85 (100,0\%)     & {[}3,4{]}   & 0,36      & 96,5\%  & 1 & 0,23     \\
        14 & 509 (100,0\%)    & {[}3,5{]}   & 3,35      & 74,9\%  & 1 & 0,81     \\
        16 & 4060 (100,0\%)   & {[}4,6{]}   & 113,39    & 51,1\%  & 2 & 8,26     \\
        18 & 41301 (100,0\%)  & {[}4,6{]}   & 5659,99   & 64,9\%  & 2 & 133,87   \\
        20 & 195284 (amostras) & {[}4,7{]}   & 116893,46 & 44,6\%  & 2 & 2584,08  \\
        22 & 217 (amostras)    & 7  & 610,54    & 49,8\%  & 1 & 47952,09 \\
        26 & 33 (amostras)     & 9   & 1404,18   & 0,0\%   & 2 & 4,68     \\
        28 & 9 (amostras)      & {[}9,10{]}  & 1508,25   & 0,0\%   & 2 & 4,98     \\
        32 & 180 (amostras)    & 11 & 172318,34 & 0,0\%   & 2 & 5,76  
    \end{tabular}%
}
\captionof{table}{Resultados c(G) grafos cúbicos.}
\label{tab-comparativo-cubic}
%\end{minipage}
\columnbreak
\resizebox{.35\textwidth}{!}{%
    \begin{tikzpicture}
        \begin{axis}
            [xlabel={Nº de Vértices}, ylabel={Nº de Carathéodory}, 
             legend pos=north west,clip=false,axis lines=left,
            ymin=1, ymax=12, ytick={2,3,4,5,6,7,8,9,10,11,12}, xmax=33,
             xtick={4,6,8,10,12,14,16,18,20,22,24,26,28,30,32}]
             \addplot[name path=maxi,color=red,mark=none] coordinates {
                (4,2) (6,2) (8,3)
                (10,3) (12,3) (14,3)
                (16,4) (18,4)
             };
             \addplot[name path=mini,color=blue,mark=none] coordinates {
                (4,2) (6,2) (8,3)
                (10,4) (12,4) (14,5)
                (16,6) (18,6) 
             };
             \addplot[fill=blue, fill opacity=0.3] fill between[of=maxi and mini];

             \addlegendentry{mínimo}
             \addlegendentry{máximo}
             \addplot[dotted, name path=miniaprox,color=red] coordinates {
               (18,4) (20,4) (22,7) (26,9) (28,9) (32,11)
             };
            \addplot[dotted, name path=maxaprox,color=blue] coordinates {
               (18,6) (20,7) (22,7) (26,9) (28,10) (32,11)
             };
        \end{axis}
    \end{tikzpicture}%
}
\end{multicols}

Conforme visto na revisão bibliográfica foi demonstrado 
a existência de um algoritmo polinomial para o número envoltório $P_3$ em grafos cúbicos.
Analisando todo os grafos até 20 vértices para o número envoltório. 
Temos um limite superior próximo a $h(G) \le \lceil\frac{n}{3}\rceil$.
Porém existe apenas um grafo de 18 vértices, que ultrapassa esse limite.
Precisando assim de mais resultados e uma melhor analise para conjecturar o limite dessa classe.
E quem sabe assim melhorar o resultado da literatura.
Na Tabela~\ref{tab-comparativo-hn-cubic} temos os resultados desse parâmetro.


\begin{multicols}{2}
%\captionof{figure}{Comportamento h(G) C.}
%\label{graf-comportamento-hn-xx}
%\centering
\captionof{table}{Resultados h(G) grafos cúbicos}
\label{tab-comparativo-hn-cubic}
\raggedleft
%\centering
\resizebox{.60\textwidth}{!}{%
    \begin{tabular}{r|r|r|r|r|r|r}
    \textbf{} & \multicolumn{3}{c|}{\textbf{\begin{tabular}[c]{@{}c@{}}Algoritmo 3.\ref{alg:numero-envoltoria-p3} \\ $NumeroEnvoltorio$\end{tabular}}} 
              & \multicolumn{3}{c}{\textbf{\begin{tabular}[c]{@{}c@{}}Algoritmo 3.\ref{alg:aproximativo-numero-envoltoria-p3} \\ $AproxNEnvoltoria$\end{tabular}}} \\ \hline
    \textbf{Nº Vert.} & \textbf{Quantidade} & \textbf{h(G)} & \textbf{T(s)} & \textbf{Acertos} & \textbf{max$\{\Delta(r)\}$} & \textbf{T(s)} \\ \hline
        4,6  & 3 (100\%)       & 2 & 0         & 100,0\% & 0  & 0       \\
        %6  & 2 (100\%)       & 2 & 0         & 100,0\% & 0  & 0       \\
        8  & 5 (100\%)       & 3 & 0         & 100,0\% & 0  & 0,01    \\
        10 & 19 (100\%)      & {[}3,4{]} & 0         & 36,8\%  & 1  & 0,01    \\
        12 & 85 (100\%)      & 4 & 0,05      & 95,3\%  & 1  & 0,03    \\
        14 & 509 (100\%)     & {[}4,5{]} & 0,24      & 3,7\%   & 2  & 0,15    \\
        16 & 4060 (100\%)    & {[}5,6{]} & 8,33      & 77,1\%  & 1  & 0,79    \\
        18 & 41301 (100\%)   & {[}5,7{]} & 174,36    & 2,2\%   & 2  & 10,27   \\
        20 & 510489 (100\%)  & {[}6,7{]} & 12544,54  & 61,8\%  & 2  & 203,68  \\
        22 & 3408042 (50\%) & {[}6,8{]} & 109776,14 & 42,8\%  & 3  & 4393,67 \\
        24 & 4993 (amostras)    & {[}7,7{]} & 1096,11   & --      & -- & --     
    \end{tabular}%
}
%\end{minipage}
\columnbreak
\resizebox{.35\textwidth}{!}{%
    \begin{tikzpicture}
        \begin{axis}
            [xlabel={Nº de Vértices}, ylabel={Nº Envoltório}, 
             legend pos=north west,clip=false,axis lines=left,
             ymin=1, ymax=9, ytick={2,3,4,5,6,7,8,9}, xmax=25,
             xtick={4,6,8,10,12,14,16,18,20,22,24}]
             \addplot[name path=maxi,color=red,mark=none] coordinates {
                (4,2) (6,2) (8,3)
                (10,3) (12,4) (14,4)
                (16,5) (18,5) (20,6) 
             };
             \addplot[name path=mini,color=blue,mark=none] coordinates {
                (4,2) (6,2) (8,3)
                (10,4) (12,4) (14,5)
                (16,6) (18,7) (20,7) 
             };
             \addplot[fill=blue, fill opacity=0.3] fill between[of=maxi and mini];

             \addlegendentry{mínimo}
             \addlegendentry{máximo}
             \addplot[dotted, name path=miniaprox,color=red] coordinates {
                (20,6) (22,6) (24,7)
             };
            \addplot[dotted, name path=maxaprox,color=blue] coordinates {
                (20,7) (22,8) (24,7)
             };
        \end{axis}
    \end{tikzpicture}%
}
\end{multicols}

Vemos também que a partir de 14 vértices, 
os algoritmos aproximativos não demonstraram bons resultados,
ficando com acertos exatos com média a baixo de 50\%.
A inferência de conjecturas também não foi possível em analise previa.
Com uma quantidade maior de resultados e análise dos poucos casos divergentes,
talvez seja possível estabelecer uma conjectura de limite $h(G) \le \lceil\frac{n}{3}\rceil +1$.

A observação dos resultados do número de Carathéodory
visto no gráfico da Tabela~\ref{tab-comparativo-strongly-vertex-transitive}.
Indica um crescimento ilimitado e não previsível desse parâmetro.

\begin{multicols}{2}
%\captionof{figure}{Comportamento c(G) VT.}
%\label{graf-comportamento-vertex-transitive}
%\centering
\raggedleft

\resizebox{.60\textwidth}{!}{%
    \begin{tabular}{r|r|r|r|r|r|r}
    \textbf{} & \multicolumn{3}{c|}{\textbf{\begin{tabular}[c]{@{}c@{}}Algoritmo 3.\ref{alg:numero-caratheodory-p3} \\ $NumeroCaratheodory$\end{tabular}}} 
              & \multicolumn{3}{c}{\textbf{\begin{tabular}[c]{@{}c@{}}Algoritmo 3.\ref{alg:aproximativo-numero-caratheodory-p3} \\ $AproxNCaratheodory$\end{tabular}}} \\ \hline
    \textbf{Nº Vert.} & \textbf{Quantidade} & \textbf{c(G)} & \textbf{T(s)} & \textbf{Acertos} & \textbf{max$\{\Delta(r)\}$} & \textbf{T(s)} \\ \hline
    3-7  & 13 (100\%)    & {[}2,2{]} & 0      & 100,0\% & 0 & 0      \\
%    4  & 2    & {[}2,2{]} & 0      & 100,0\% & 0 & 0,01   \\
%    5  & 2    & {[}2,2{]} & 0      & 100,0\% & 0 & 0      \\
%    6  & 5    & {[}2,2{]} & 0,01   & 100,0\% & 0 & 0,02   \\
%    7  & 3    & {[}2,2{]} & 0      & 100,0\% & 0 & 0,02   \\
    8  & 10 (100\%)    & {[}2,3{]} & 0,03   & 100,0\% & 0 & 0,03   \\
    9  & 7 (100\%)    & {[}2,2{]} & 0,02   & 100,0\% & 0 & 0,02   \\
    10 & 18 (100\%)   & {[}2,4{]} & 0,11   & 94,4\%  & 1 & 0,13   \\
    11 & 7 (100\%)    & {[}2,3{]} & 0,04   & 100,0\% & 0 & 0,07   \\
    12 & 64 (100\%)   & {[}2,4{]} & 0,35   & 98,4\%  & 1 & 0,54   \\
    13 & 13 (100\%)   & {[}2,3{]} & 0,11   & 100,0\% & 0 & 0,07   \\
    14 & 51 (100\%)   & {[}2,5{]} & 0,29   & 98,0\%  & 1 & 0,51   \\
    15 & 44 (100\%)   & {[}2,4{]} & 0,53   & 95,5\%  & 1 & 0,22   \\
    16 & 272 (100\%)  & {[}2,6{]} & 2,93   & 94,1\%  & 1 & 1,61   \\
    17 & 35 (100\%)   & {[}2,3{]} & 0,42   & 100,0\% & 0 & 0,42   \\
    18 & 365 (100\%)  & {[}2,6{]} & 7,53   & 97,5\%  & 1 & 2,79   \\
    19 & 59 (100\%)   & {[}2,4{]} & 1,13   & 98,3\%  & 1 & 0,55   \\
    20 & 1190 (100\%) & {[}2,7{]} & 35,2   & 98,1\%  & 1 & 8,87   \\
    21 & 235 (100\%)  & {[}2,4{]} & 14,01  & 97,9\%  & 1 & 1,74   \\
    22 & 807 (100\%)  & {[}2,8{]} & 43,69  & 99,1\%  & 1 & 6,46   \\
    23 & 187 (100\%)  & {[}2,4{]} & 11,24  & 99,5\%  & 1 & 1,92   \\
    24 & 15422 (100\%) & {[}2,8{]} & 1197,1 & 10,5\%  & 1 & 181,91 \\
    25 & 461 (100\%)  & {[}2,5{]} & 110,82 & 98,9\%  & 1 & 6,29
    \end{tabular}%
}
\captionof{table}{Resultados c(G) grafos vértice-transitivo.}
\label{tab-comparativo-strongly-vertex-transitive}
%\end{minipage}
\columnbreak
%\centering
\resizebox{.35\textwidth}{!}{%
    \begin{tikzpicture}
        \begin{axis}
            [xlabel={Nº de Vértices}, ylabel={Nº de Carathéodory}, 
             legend pos=north west,clip=false,axis lines=left,
             ymin=1, ymax=9, ytick={2,3,4,5,6,7,8,9}, xmax=28,
             xtick={3,4,5,6,7,8,9,10,12,14,16,18,20,22,24,26,28}]
             \addplot[name path=maxi,color=red,mark=none] coordinates {
                (3,2) (25,2) 
             };
             \addlegendentry{mínimo}
             \addplot[name path=mini,color=blue,mark=none] coordinates {
                (3,2) (7,2) (8,3)(9,2)
                (10,4) (11,3) (12,4) (13,3)
                (14,5) (16,6) (17,3) (18,6) 
                (19,4) (20,7) (21,4) (22,8)
                (23,4) (24,8) (25,5) 
             };
             \addlegendentry{máximo}
             \addplot[fill=blue, fill opacity=0.3] fill between[of=maxi and mini];
        \end{axis}
    \end{tikzpicture}%
}
\end{multicols}